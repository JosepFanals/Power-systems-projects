
\epigraph{Let every action aim solely at the common good}{\textit{Marcus Aurelius}}

The present project has unveiled a framework of operation that combines scalable mechanisms for both computing the power flow and allowing transactions between nodes in the grid. This approach is likely to establish the necessary bases to move from the current power system towards a decentralized grid where nodes in the grid are motivated to actively participate in improving the operating state of the system and are rewarded for doing so. 

The cooperation of users is first of all achieved by deploying smart contracts on the blockchain, which allow users to securely trade energy. A critical decision regarding the chosen approach has to do with opting for the Matic Mainnet, a network that allows transacting by paying fees which become several orders of magnitude cheaper than in case they were deployed directly on Ethereum. The code has been written in Solidity and called from Python to offer more flexibility.

Regarding the calculation of the optimal power flow, the PGD methodology has been mathematically derived as a tool to solve multidimensional power flow problems involving not only position and time dimensions but also covering variations in the parameters of the system. In the particular scenario that has been analyzed in the case study section, the PV power generation could vary between a wide range of values. The PGD obtained the optimal power so that losses were minimized. We believe the PGD is a transformative idea that can offer operators a deeper understanding of the system in shorter time frames.

Even though the project has many difficulties to overcome, it presents an original idea that could become a reality. We are aware of the potential and the future of blockchain technology, and also the need to run fast power flow calculations, which the PGD handles remarkably well. Combining this with almost negligible operating costs, we believe that this work is just the start of something bigger.



