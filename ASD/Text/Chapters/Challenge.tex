Concerning the state of the art, we have identified three ideas that remained either uncovered or that offer the possibility of being extended:
\begin{itemize}
    \item The deployment of smart contracts with low transaction costs (also referred to gas fees on the Ethereum network) would induce the participation of prosumers in a blockchain-based electric system.
    \item The utilization of an efficient methodology to solve the time series power flow. This is particularly imperative in the case of real-time markets. 
    \item The coupling of the two aforementioned ideas in a single program that combines the best of both worlds.
\end{itemize}
In more detail, smart contracts deployed on the Ethereum network imply highly variable transaction costs. For example, the fees can more than triple in a weekly time frame \cite{eth_gas}. This would not be a problem if these transaction costs were insignificant in comparison to the cost of electricity. However, at the time of writing this, a single transaction is charged at around 100~gwei, which in monetary terms translates into 4 \euro \  approximately \cite{eth_price}. This exorbitant value would be unacceptable in a real-time market, not only because of the price fluctuations but also because the transaction cost could exceed the price of the traded electricity. Matic, on the other hand, becomes much more suitable for real-time energy trading. The average time to form a block is close to 2 seconds (contrarily to 13 s for Ethereum), and the associated transaction cost is around 0.0001 \euro \ \cite{eth_price2, matic_expl}.

Regarding the power flow, traditional and current methods solve it at every time step. It does not matter \textit{a priori} if a load just changes slightly its value; to ensure the obtention of a satisfactory result, the power flow has to be computed again and again. From the outside, this seems a waste of computational effort because the structure of the system remains the same. Therefore, the question to formulate is: could we retain the general properties of the system to solve the system more efficiently? In a nutshell, the PGD formulation answers this issue. It generates a fast and precise enough solution to the power flow problem from where the optimal situation can be easily derived. 

Finally, both the smart contracts and the PGD are implementable separately. Nevertheless, we consider that the most favorable outcomes lie in the intersection, i.e., smart contracts can benefit from an efficient calculation of the power flow problem and vice versa. As the Greek philosopher Aristotle once stated, "the whole is greater than the sum of its parts". We stand by this, and so our mission is to now show why this is the case.

% put a graph of the costs of the Ethereum network??