During the last decades, power systems have evolved towards decentralized schemes with a larger contribution of renewables. This transformation has caused a reduction in the generation from traditional power plants, which in turn implies a decrease in the associated carbon footprint. However, this does not come freely. There are many challenges to face, such as controlling the upcoming bi-directional power flow, empowering prosumers, managing huge volumes of data, among others.

This work attempts to focus on two imperative issues. One of them has to do with the physical transmission of electricity, namely the power flow problem. As far as we are aware, the current iterative methods are not scalable, in the sense that the required computation time grows with the dimensions of the system. One dimension corresponds to the number of buses, while a second one could be related to the discretized time frames, another to the ranges of powers of both loads and generators, etc. Magnifying one dimension by a certain ratio causes the total computational effort to increase by that factor at least. This ever-growing difficulty is especially acute when solving the optimal power flow, as input data may take a wide range of values. To combat such complications, we implement the Proper Generalized Decomposition (PGD) methodology. In short, it allows to compute the solution to the multi-dimensional power flow much more efficiently, and thus, more insight can be gained from it in the same amount of time. 

The second issue considered in this project has to do with enhancing prosumers' participation and placing more power on their hands as well. We propose a flow of information based on smart contracts in order to make the transition towards a semi-decentralized network. Prosumers' participation in generating and consuming power will be stimulated so as to reach a near-optimal state, previously computed with the PGD tool. Despite the attractiveness found in that idea, the transaction costs (the so-called gas fees) linked to the Ethereum blockchain are likely to be exceedingly high. Therefore, we deploy the smart contracts on a level 2 layer, such as Matic. This project allows transactions to become several orders of magnitude cheaper than what they typically are. The code will still be programmed in Solidity and it will be able to establish communication with the Python code that solves the optimal operating point of the system. 

This work is structured as follows: Chapter \ref{State} describes the current state of the art related to the methodology to solve the optimal power flow as well as to blockchain and in particular to smart contracts; Chapter \ref{Challenge} presents the challenges addressed that are detailed further on; Chapters \ref{Contracts} and \ref{PGD} constitute the main work of the project, as they focus on the smart contracts and the PGD respectively; Chapter \ref{Case} shows the practical implementation of the covered tools; Chapter \ref{Implementation} discusses the economical and business-related aspects, together with a potential integration into EDP's system; Chapter \ref{Plan} presents an implementation plan to aid at bringing the project to reality; finally, Chapter \ref{Conclusions} articulates the conclusions.

