It is a well-known fact that the power flow in power systems is represented by non-linear equations. Multiple solvers have been applied during these past decades. Some of them have been intentionally created for such purpose, while others are based on common numerical methods, being the Newton-Raphson the most popular. At its core, all these solvers are intented to be used with deterministic input data, that is, they leave no room for variation. In case an input changes, the power flow has to be recalculated.

With this conventional approach, given a power system where there are $m$~uncertain parameters and each of them takes $n$~values, the solver would have to be called $n^m$~times in order to extract all solutions. Thus, an increase in the number of parameters involves a meteoric rise in the computational effort, which is most likely unjustified. 

The probabilistic power flow addresses the uncertainty in the parameters' values. It aims at generating the probability distribution of a grid state (e.g., the voltage magnitude at a given bus) as a function of the input varying data (such as the active power provided by a generator). It was first formulated by Borkowska back in 1974 \cite{borkowska1974}. Nowadays it has gained attraction in the realm of renewable energy sources due to their relative unpredictability. In particular, it has been employed for photovoltaics and wind power, which rely on Gaussian and Weibull distributions respectively \cite{li2019, morales2010, fan2012}. Moreover, the impact of the electric vehicle charging demand on the grid has been assessed with this technique \cite{li2012}. However, using the probabilistic power flow does not allow to know with certainty a grid state considering deterministic input data. 

The parametric power flow ideally provides a solution to the aforementioned challenge. Instead of dealing with distributions, it generates a closed-form expression that relates the input data (called parameters from now on) with the state of interest. It sacrificies a bit of precision in order to obtain a satisfactory approximation, much more computationally efficient than solving the $n^m$~cases. It has been implemented in conjunction with an optimal power flow formulation \cite{almeida1994}, yet for the most part it has remained an unexplored methodology. García-Blanco et al. have formulated the so-called Proper Generalized Decomposition, which manages to obtain all $n^m$ solutions reasonably fast and accurately \cite{garcia2017, blanco2017}. Nevertheless, the methodology is not scalable. It is tied to a traditional power system formed by PQ buses and a single slack bus. There are workarounds to add PV buses but they are prone to cause divergence. Thus, as power systems tend to incorporate controllable power electronic devices, there is little relevance in such technique. A much more appealing approach, formulated by Shen et al., deals with expressing a state as a function of the parameters \cite{shen2020}. This is the central reference of this work. 

This monograph presents a detailed formulation of the parametric power flow and justifies how the dimensions can be reduced. Then, a basic 5-bus system is tackled to exemplify the methodology. Some thoughts on the implications of this technique are also verbalized. Finally, conclusions are extracted along with the conceptualization of potential research ideas. 
